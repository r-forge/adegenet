\documentclass{article}
% \VignettePackage{adegenet-genomics}
% \VignetteIndexEntry{Analysing genome-wide SNP data using adegenet}

\usepackage{graphicx}
\usepackage[colorlinks=true,urlcolor=blue]{hyperref}
\usepackage{array}
\usepackage{color}

\usepackage[utf8]{inputenc} % for UTF-8/single quotes from sQuote()
\newcommand{\code}[1]{{{\tt #1}}}
\title{Analysing genomic-wide SNP data using  \textit{adegenet} 1.3-0}
\author{Thibaut Jombart}
\date{\today}




\sloppy
\hyphenpenalty 10000


\usepackage{Sweave}
\begin{document}





\definecolor{Soutput}{rgb}{0,0,0.56}
\definecolor{Sinput}{rgb}{0.56,0,0}
\DefineVerbatimEnvironment{Sinput}{Verbatim}
{formatcom={\color{Sinput}},fontsize=\footnotesize, baselinestretch=0.75}
\DefineVerbatimEnvironment{Soutput}{Verbatim}
{formatcom={\color{Soutput}},fontsize=\footnotesize, baselinestretch=0.75}

\color{black}

\maketitle

\begin{abstract}
  Genome-wide SNP data can quickly be challenging to analyse using standard
  computer. The package \textit{adegenet} \cite{tjart05} for the R software \cite{np145}
  implements representation of these data with unprecedented efficiency
  using the classes \texttt{SNPbin} and \texttt{genlight}, which can require up to 60 times less RAM than usual
  representation using allele frequencies.
  This vignette introduces these classes and illustrates how these objects can be handled and
  analyzed in R.
  It also introduces more advanced features of an API in C language which may be useful to develop
  new method based on these objects.
\end{abstract}

\newpage

\tableofcontents


\newpage
%%%%%%%%%%%%%%%%
%%%%%%%%%%%%%%%%
\section{Introduction}
%%%%%%%%%%%%%%%%
%%%%%%%%%%%%%%%%
Modern sequencing technologies now make complete genomes more widely accessible.
The subsequent amounts of genetic data pose challenges in terms of storing and handling the data,
making former tools developed for classical genetic markers such as microsatellite impracticable using
standard computers.
Adegenet has developed new object classes dedicated to handling genome-wide polymorphism (SNPs) with
minimum rapid access memory (RAM) requirements.
\\

Two new formal classes have been implemented: \texttt{SNPbin}, used to store genome-wide SNPs for
one individual, and \texttt{genlight}, which stored the same information for multiple individuals.
Information represented this way is binary: only biallelic SNPs can be stored and analyzed using these classes.
However, these objects are otherwise very flexible, and can incorporate different levels of ploidy
across individuals within a single dataset.
In this vignette, we present these object classes and show how their content can be further handled and
content analyzed.





%%%%%%%%%%%%%%%%
%%%%%%%%%%%%%%%%
\section{Classes of objects}
%%%%%%%%%%%%%%%%
%%%%%%%%%%%%%%%%

%%%%%%%%%%%%%%%%
\subsection{\code{SNPbin}: storage of single genomes}
%%%%%%%%%%%%%%%%
The class \texttt{SNPbin} is the core representation of biallelic SNPs which allows to represent
data with unprecedented efficiency.
The essential idea is to code binary SNPs not as integers, but as bits. This operation is tricky in
R as there is no handling of bits, only bytes -- series of 8 bits. However, the class
\texttt{SNPbin} handles this transparently using sub-rountines in C language.
Considerable efforts have been made so that the user does not have to dig into the complex internal
structure of the objects, and can handle \texttt{SNPbin} objects as easily as possible.
\\

Like \texttt{genind} and \texttt{genpop} objects, \texttt{SNPbin} is a formal "S4" class. The
structure of these objects is detailed in the dedicated manpage (\texttt{?SNPbin}). As all S4
objects, instances of the class \texttt{SNPbin} are composed of slots accessible using the
\texttt{@} operator. This content is generic (it is the same for all instances of the class), and returned by:
\begin{Schunk}
\begin{Sinput}
> library(adegenet)
> getClassDef("SNPbin")
\end{Sinput}
\begin{Soutput}
Class "SNPbin" [package "adegenet"]

Slots:
                                                             
Name:         snp      n.loc    NA.posi      label     ploidy
Class:       list    integer    integer charOrNULL    integer
\end{Soutput}
\end{Schunk}

The slots respectively contain:
\begin{itemize}
  \item \texttt{snp}: SNP data with specific internal coding.
  \item \texttt{n.loc}: the number of SNPs stored in the object.
  \item \texttt{NA.posi}: position of the missing data (NAs).
  \item \texttt{label}: an optional label for the individual.
  \item \texttt{ploidy}: the ploidy level of the genome.
\end{itemize}

New objects are created using \texttt{new}, with these slots as arguments.
If no argument is provided, an empty object is created:
\begin{Schunk}
\begin{Sinput}
> new("SNPbin")
\end{Sinput}
\begin{Soutput}
 === S4 class SNPbin ===
 0 SNPs coded as bits
 Ploidy: NA
 2 (Inf %) missing data
\end{Soutput}
\end{Schunk}
In practice, only the \texttt{snp} information and possibly the ploidy has to be provided; various
formats are accepted for the \texttt{snp} component, but the simplest is a vector of integers (or
numeric) indicating the number of second allele at each locus.
The argument \texttt{snp}, if provided alone, does not have to be named:
\begin{Schunk}
\begin{Sinput}
> x <- new("SNPbin", c(0, 1, 1, 2, 0, 0, 1))
> x
\end{Sinput}
\begin{Soutput}
 === S4 class SNPbin ===
 7 SNPs coded as bits
 Ploidy: 2
 0 (0 %) missing data
\end{Soutput}
\end{Schunk}

If not provided, the ploidy is detected from the data and determined as the largest number in the
input vector. Obviously, in many cases this will not be adequate, but ploidy can always be rectified
afterwards; for instance:
\begin{Schunk}
\begin{Sinput}
> x
\end{Sinput}
\begin{Soutput}
 === S4 class SNPbin ===
 7 SNPs coded as bits
 Ploidy: 2
 0 (0 %) missing data
\end{Soutput}
\begin{Sinput}
> ploidy(x) <- 3
> x
\end{Sinput}
\begin{Soutput}
 === S4 class SNPbin ===
 7 SNPs coded as bits
 Ploidy: 3
 0 (0 %) missing data
\end{Soutput}
\end{Schunk}

\noindent The internal coding of the objects is cryptic, and not meant to be accessed directly:
\begin{Schunk}
\begin{Sinput}
> x@snp
\end{Sinput}
\begin{Soutput}
[[1]]
[1] 08

[[2]]
[1] 4e
\end{Soutput}
\end{Schunk}
Fortunately, data are easily converted back into integers:
\begin{Schunk}
\begin{Sinput}
> as.integer(x)
\end{Sinput}
\begin{Soutput}
[1] 0 1 1 2 0 0 1
\end{Soutput}
\end{Schunk}

~\\

The main interest of this representation is its efficiency in terms of storage.
For instance:
\begin{Schunk}
\begin{Sinput}
> dat <- sample(0:1, 1e+06, replace = TRUE)
> print(object.size(dat), unit = "auto")
\end{Sinput}
\begin{Soutput}
3.8 Mb
\end{Soutput}
\begin{Sinput}
> x <- new("SNPbin", dat)
> print(object.size(x), unit = "auto")
\end{Sinput}
\begin{Soutput}
122.8 Kb
\end{Soutput}
\end{Schunk}
here, we converted a million SNPs into a \texttt{SNPbin} object, which turns out to be
32 smaller than the original data.
However, the information in \texttt{dat} and \texttt{x} is strictly identical:
\begin{Schunk}
\begin{Sinput}
> identical(as.integer(x), dat)
\end{Sinput}
\begin{Soutput}
[1] TRUE
\end{Soutput}
\end{Schunk}
The advantage of this storage is therefore being extremely compact, and allowing to analyse big
datasets using standard computers.
%% Obviously, usual computations demand data to be at one moment coded as numeric values (as opposed to bits).
%% However in most cases, we can proceed by only converting one or two genomes back to numeric values
%% at a time, therefore keeping RAM requirements low, albeit at a possible increase in computational time.
%% This however is minimized by two ways: i) conversion routines are optimized for speed using C code
%% ii) smaller objects are handled, therefore decreasing the possibly high computational time taken by memory allocation.
\\

While \texttt{SNPbin} objects are the very mean by which we store data efficiently, in practice
we need to analyze several genomes at a time.
This is made possible by the class \texttt{genlight}, which relies on \texttt{SNPbin} but allows for
storing data from several genomes at a time.




%%%%%%%%%%%%%%%%
\subsection{\code{genlight}: storage of multiple genomes}
%%%%%%%%%%%%%%%%

Like \texttt{SNPbin}, \texttt{genlight} is a formal S4 class.
The slots of instances of this class are described by:
\begin{Schunk}
\begin{Sinput}
> getClassDef("genlight")
\end{Sinput}
\begin{Soutput}
Class "genlight" [package "adegenet"]

Slots:
                                                                       
Name:           gen        n.loc    ind.names    loc.names      loc.all
Class:         list      integer   charOrNULL   charOrNULL   charOrNULL
                                                                       
Name:    chromosome     position       ploidy          pop        other
Class: factorOrNULL    intOrNULL    intOrNULL factorOrNULL         list
\end{Soutput}
\end{Schunk}
As it can be seen, these objects allow for storing more information in addition to vectors of SNP frequencies.
More precisely, their content is (see \texttt{?genlight} for more details):
\begin{itemize}
  \item \texttt{gen}: SNP data for different individuals, each stored as a \texttt{SNPbin}; loci
    have to be identical across all individuals.
  \item \texttt{n.loc}: the number of SNPs stored in the object.
  \item \texttt{ind.names}: (optional) labels for the individuals.
  \item \texttt{loc.names}: (optional) labels for the loci.
  \item \texttt{loc.all}: (optional) alleles of the loci separated by '/' (e.g. 'a/t', 'g/c', etc.).
  \item \texttt{chromosome}: (optional) a factor indicating the chromosome to which the SNPs belong.
  \item \texttt{position}: (optional) the position of each SNPs in their chromosome.
  \item \texttt{ploidy}: (optional) the ploidy of each individual.
  \item \texttt{pop}: (optional) a factor grouping individuals into 'populations'.
  \item \texttt{other}: (optional) a list containing any supplementary information to be stored with
    the data.
\end{itemize}

\noindent Like \texttt{SNPbin} object, \texttt{genlight} object are created using the constructor \texttt{new},
providing content for the slots above as arguments.
When none is provided, an empty object is created:
\begin{Schunk}
\begin{Sinput}
> new("genlight")
\end{Sinput}
\begin{Soutput}
 === S4 class genlight ===
 0 genotypes,  0 binary SNPs
\end{Soutput}
\end{Schunk}
The most important information to provide is obviously the genotypes (argument \texttt{gen}); these
can be provided as:
\begin{itemize}
\item a \texttt{list} of integer vectors representing the number of second allele at each locus.
\item a \texttt{matrix} / \texttt{data.frame} of integers, with individuals in rows and SNPs in columns.
\item a list of \texttt{SNPbin} objects.
\end{itemize}

Ploidy has to be consistent across loci for a given individual, but individuals do not have to have
the same ploidy, so that it is possible to have hapoid,
diploid, and tetraploid individuals in the same dataset; for instance:
\begin{Schunk}
\begin{Sinput}
> x <- new("genlight", list(indiv1 = c(1, 1, 0, 1, 1, 0), indiv2 = c(2, 
+     1, 1, 0, 0, 0), toto = c(2, 2, 0, 0, 4, 4)))
> x
\end{Sinput}
\begin{Soutput}
 === S4 class genlight ===
 3 genotypes,  6 binary SNPs
 Ploidy statistics (min/median/max): 1 / 2 / 4
 0 (0 %) missing data
\end{Soutput}
\begin{Sinput}
> ploidy(x)
\end{Sinput}
\begin{Soutput}
indiv1 indiv2   toto 
     1      2      4 
\end{Soutput}
\end{Schunk}

As for \texttt{SNPbin}, \texttt{genlight} objects can be converted back to integers vectors, stored
as matrices or lists:
\begin{Schunk}
\begin{Sinput}
> as.list(x)
\end{Sinput}
\begin{Soutput}
$indiv1
[1] 1 1 0 1 1 0

$indiv2
[1] 2 1 1 0 0 0

$toto
[1] 2 2 0 0 4 4
\end{Soutput}
\begin{Sinput}
> as.matrix(x)
\end{Sinput}
\begin{Soutput}
       [,1] [,2] [,3] [,4] [,5] [,6]
indiv1    1    1    0    1    1    0
indiv2    2    1    1    0    0    0
toto      2    2    0    0    4    4
\end{Soutput}
\end{Schunk}

\noindent In practice, \texttt{genlight} objects can be handled as if they were matrices of integers
as the one above returned by \texttt{as.matrix}.
However, they offer the advantage of efficient storage of the information; for instance, we can
simulate 50 individuals typed for 1,00,000 SNPs each (including occasional NAs):
\begin{Schunk}
\begin{Sinput}
> dat <- lapply(1:50, function(i) sample(c(0, 1, NA), 1e+06, prob = c(0.5, 
+     0.499, 0.001), replace = TRUE))
> names(dat) <- paste("indiv", 1:length(dat))
> print(object.size(dat), unit = "auto")
\end{Sinput}
\begin{Soutput}
381.5 Mb
\end{Soutput}
\begin{Sinput}
> x <- new("genlight", dat)
> print(object.size(x), unit = "auto")
\end{Sinput}
\begin{Soutput}
6.2 Mb
\end{Soutput}
\begin{Sinput}
> object.size(dat)/object.size(x)
\end{Sinput}
\begin{Soutput}
61.6512065339821 bytes
\end{Soutput}
\end{Schunk}
here again, the storage if the data is much more efficient in \texttt{genlight} than using integers: converted data occupy
62 times less memory than the original data.
\\

The advantage of this storage is therefore being extremely compact, and allowing to analyse very large
datasets using standard computers.
Obviously, usual computations demand data to be at one moment coded as numeric values (as opposed to bits).
However, most usual computations can be achieved by only converting one or two genomes back to numeric values
at a time, therefore keeping RAM requirements low, albeit at a possible cost of increased computational time.
This however is minimized by three ways:
\begin{enumerate}
\item conversion routines are optimized for speed using C code.
\item using parallel computation where multicore architectures are available.
\item handling smaller objects, thereby decreasing the possibly high computational time taken by memory allocation.
\end{enumerate}

While this makes implementing methods more complicated.
In practice, routines are implemented so as to minimize
the amount of data converted back to integers, use C code where possible, and use multiple cores
if the package \textit{multicore} is installed an multiple cores are available.
Fortunately, these underlying technical issues are oblivious to the user, and one merely needs to
know how to manipulate \texttt{genlight} objects using a few key functions to be able to analyze data.






%%%%%%%%%%%%%%%%
%%%%%%%%%%%%%%%%
\section{Data handling using \texttt{genlight} objects}
%%%%%%%%%%%%%%%%
%%%%%%%%%%%%%%%%

%%%%%%%%%%%%%%%%
\subsection{Using accessors}
%%%%%%%%%%%%%%%%

In the following, we demonstrate how to manipulate and analyse \texttt{genlight} objects.
The phylosophy underlying formal (S4) classes in general, and \texttt{genlight} objects in
particular, is that internal representation of the information can be complex as long as accessing
this information is simple.
This is made possible by decoupling storage and accession: the user is not meant to access the
content of the object directly, but has to use \texttt{accessors} to retrieve or modify information.
\\

Available accessors are documented in \code{?genlight}.
Most of them are identical to accessors for \texttt{genind} and \texttt{genpop} objects, such as:
\begin{itemize}
  \item \texttt{nInd}: returns the number of individuals in the object.
  \item \texttt{nLoc}: returns the number of loci (SNPs).
  \item \texttt{indNames}$^{\dagger}$: returns/sets labels for individuals.
  \item \texttt{locNames}$^{\dagger}$: returns/sets labels for loci (SNPs).
  \item \texttt{alleles}$^{\dagger}$: returns/sets alleles.
  \item \texttt{ploidy}$^{\dagger}$: returns/sets ploidy of the individuals.
  \item \texttt{pop}$^{\dagger}$: returns/sets a factor grouping individuals.
  \item \texttt{other}$^{\dagger}$: returns/sets misc information stored as a list.
\end{itemize}
where $^{\dagger}$ indicates that a replacement method is available using \texttt{<-'}; for instance:
\begin{Schunk}
\begin{Sinput}
> dat <- lapply(1:3, function(i) sample(0:2, 10, replace = TRUE))
> dat
\end{Sinput}
\begin{Soutput}
[[1]]
 [1] 1 2 1 2 0 2 0 0 1 2

[[2]]
 [1] 2 0 2 1 0 0 0 0 2 0

[[3]]
 [1] 2 1 0 2 2 2 1 1 2 0
\end{Soutput}
\begin{Sinput}
> x <- new("genlight", dat)
> x
\end{Sinput}
\begin{Soutput}
 === S4 class genlight ===
 3 genotypes,  10 binary SNPs
 Ploidy: 2
 0 (0 %) missing data
\end{Soutput}
\begin{Sinput}
> indNames(x)
\end{Sinput}
\begin{Soutput}
NULL
\end{Soutput}
\begin{Sinput}
> indNames(x) <- paste("individual", 1:3)
> indNames(x)
\end{Sinput}
\begin{Soutput}
[1] "individual 1" "individual 2" "individual 3"
\end{Soutput}
\begin{Sinput}
> locNames(x)
\end{Sinput}
\begin{Soutput}
NULL
\end{Soutput}
\begin{Sinput}
> locNames(x) <- paste("SNP", 1:nLoc(x), sep = ".")
> as.matrix(x)
\end{Sinput}
\begin{Soutput}
             SNP.1 SNP.2 SNP.3 SNP.4 SNP.5 SNP.6 SNP.7 SNP.8 SNP.9 SNP.10
individual 1     1     2     1     2     0     2     0     0     1      2
individual 2     2     0     2     1     0     0     0     0     2      0
individual 3     2     1     0     2     2     2     1     1     2      0
\end{Soutput}
\end{Schunk}

\noindent
In addition, some specific accessors are available for \texttt{genlight} objects:
\begin{itemize}
  \item \texttt{NA.posi}: returns the position of missing values in each individual.
  \item \texttt{chromosome}$^{\dagger}$: returns/sets the chromosome of each SNP.
  \item \texttt{chr}$^{\dagger}$: same as \texttt{chromosome} --- used as a shortcut.
  \item \texttt{position}$^{\dagger}$: returns/sets the position of each SNP.
\end{itemize}


Accessors are meant to be clever about replacement, meaning that they try hard to prevent
replacement with inconsistent values. For instance, in object \texttt{x}:
\begin{Schunk}
\begin{Sinput}
> x
\end{Sinput}
\begin{Soutput}
 === S4 class genlight ===
 3 genotypes,  10 binary SNPs
 Ploidy: 2
 0 (0 %) missing data
\end{Soutput}
\end{Schunk}
if we try to set information about the chromosomes of the SNPs, the instruction:
\begin{Schunk}
\begin{Sinput}
> chr(x) <- rep("chr-1", 7)
\end{Sinput}
\end{Schunk}
will generate an error because the provided factor does not match the number of loci (10), while:
\begin{Schunk}
\begin{Sinput}
> chr(x) <- rep("chr-1", 10)
> x
\end{Sinput}
\begin{Soutput}
 === S4 class genlight ===
 3 genotypes,  10 binary SNPs
 Ploidy: 2
 0 (0 %) missing data
 @chromosome: chromosome of the SNPs
\end{Soutput}
\begin{Sinput}
> chr(x)
\end{Sinput}
\begin{Soutput}
 [1] chr-1 chr-1 chr-1 chr-1 chr-1 chr-1 chr-1 chr-1 chr-1 chr-1
Levels: chr-1
\end{Soutput}
\end{Schunk}
is a valid replacement.




%%%%%%%%%%%%%%%%
\subsection{Subsetting the data}
%%%%%%%%%%%%%%%%
\texttt{genlight} objects are meant to be handled as if they were matrices of allele numbers, as
returned by \texttt{as.matrix}.
Therefore, subsetting can be achieved using $[$ \texttt{idx.row , idx.col} $]$ where \texttt{idx.row}
and \texttt{idx.col} are indices for rows (individuals) and columns (SNPs).
For instance, using the previous toy dataset, we try a few classical subsetting for matrices:
\begin{Schunk}
\begin{Sinput}
> x
\end{Sinput}
\begin{Soutput}
 === S4 class genlight ===
 3 genotypes,  10 binary SNPs
 Ploidy: 2
 0 (0 %) missing data
 @chromosome: chromosome of the SNPs
\end{Soutput}
\begin{Sinput}
> as.matrix(x)
\end{Sinput}
\begin{Soutput}
             SNP.1 SNP.2 SNP.3 SNP.4 SNP.5 SNP.6 SNP.7 SNP.8 SNP.9 SNP.10
individual 1     1     2     1     2     0     2     0     0     1      2
individual 2     2     0     2     1     0     0     0     0     2      0
individual 3     2     1     0     2     2     2     1     1     2      0
\end{Soutput}
\begin{Sinput}
> as.matrix(x[c(1, 3), ])
\end{Sinput}
\begin{Soutput}
             SNP.1 SNP.2 SNP.3 SNP.4 SNP.5 SNP.6 SNP.7 SNP.8 SNP.9 SNP.10
individual 1     1     2     1     2     0     2     0     0     1      2
individual 3     2     1     0     2     2     2     1     1     2      0
\end{Soutput}
\begin{Sinput}
> as.matrix(x[, c(TRUE, FALSE)])
\end{Sinput}
\begin{Soutput}
     SNP.1 SNP.3 SNP.5 SNP.7 SNP.9
[1,]     1     1     0     0     1
[2,]     2     2     0     0     2
[3,]     2     0     2     1     2
\end{Soutput}
\begin{Sinput}
> as.matrix(x[1:2, c(1, 1, 1, 2, 2, 2, 3, 3, 3)])
\end{Sinput}
\begin{Soutput}
     SNP.1 SNP.1 SNP.1 SNP.2 SNP.2 SNP.2 SNP.3 SNP.3 SNP.3
[1,]     1     1     1     2     2     2     1     1     1
[2,]     2     2     2     0     0     0     2     2     2
\end{Soutput}
\end{Schunk}


Moreover, one can split data into blocks of SNPs using \texttt{seploc}.
This can be achieved by specifying either a number of blocks (argument \texttt{n.block}) or the size
of the blocks (argument \texttt{block.size}). The function also allows for randomizing the
distribution of the SNPs in the blocks (argument \texttt{random=TRUE}), which is especially useful
to replace computations that cannot be achieved on the whole dataset with parallelized computations performed on random blocks.
For instance:
\begin{Schunk}
\begin{Sinput}
> x
\end{Sinput}
\begin{Soutput}
 === S4 class genlight ===
 3 genotypes,  10 binary SNPs
 Ploidy: 2
 0 (0 %) missing data
 @chromosome: chromosome of the SNPs
\end{Soutput}
\begin{Sinput}
> as.matrix(x)
\end{Sinput}
\begin{Soutput}
             SNP.1 SNP.2 SNP.3 SNP.4 SNP.5 SNP.6 SNP.7 SNP.8 SNP.9 SNP.10
individual 1     1     2     1     2     0     2     0     0     1      2
individual 2     2     0     2     1     0     0     0     0     2      0
individual 3     2     1     0     2     2     2     1     1     2      0
\end{Soutput}
\begin{Sinput}
> seploc(x, n.block = 2)
\end{Sinput}
\begin{Soutput}
$block.1
 === S4 class genlight ===
 3 genotypes,  5 binary SNPs
 Ploidy: 2
 0 (0 %) missing data
 @chromosome: chromosome of the SNPs

$block.2
 === S4 class genlight ===
 3 genotypes,  5 binary SNPs
 Ploidy: 2
 0 (0 %) missing data
 @chromosome: chromosome of the SNPs
\end{Soutput}
\begin{Sinput}
> lapply(seploc(x, n.block = 2), as.matrix)
\end{Sinput}
\begin{Soutput}
$block.1
     SNP.1 SNP.3 SNP.2 SNP.5 SNP.4
[1,]     1     1     2     0     2
[2,]     2     2     0     0     1
[3,]     2     0     1     2     2

$block.2
     SNP.8 SNP.10 SNP.9 SNP.7 SNP.6
[1,]     0      2     1     0     2
[2,]     0      0     2     0     0
[3,]     1      0     2     1     2
\end{Soutput}
\end{Schunk}
splits the data into two blocks of contiguous SNPs, while:
\begin{Schunk}
\begin{Sinput}
> lapply(seploc(x, n.block = 2, random = TRUE), as.matrix)
\end{Sinput}
\begin{Soutput}
$block.1
     SNP.3 SNP.1 SNP.2 SNP.5 SNP.10
[1,]     1     1     2     0      2
[2,]     2     2     0     0      0
[3,]     0     2     1     2      0

$block.2
     SNP.8 SNP.4 SNP.6 SNP.7 SNP.9
[1,]     0     2     2     0     1
[2,]     0     1     0     0     2
[3,]     1     2     2     1     2
\end{Soutput}
\end{Schunk}
generates blocks of randomly selected SNPs.



%%%%%%%%%%%%%%%%
\subsection{Data conversions}
%%%%%%%%%%%%%%%%



% % % % % % % % % % % % %
\subsubsection{The \texttt{.snp} format}
% % % % % % % % % % % % %

% % % % % % % % % % % % %
\subsubsection{Importing data from PLINK}
% % % % % % % % % % % % %

Genome-wide SNP data are most often stored as text files with particular formats, with format of the
software PLINK being the most popular.
Data with PLINK format can be read into \texttt{genlight} objects using \texttt{read.PLINK}.
This function requires the data to be saved in PLINK using the `-recodeA` option (see details
section in \texttt{?read.PLINK}).

\texttt{read.PLINK} has the advantage of being able to read the data by chunks of a few individuals
(up to a single one at a time), which minimizes the amount of memory needed to read information
before its conversion to \texttt{genlight}; however, using more chunks also means more computational
time, since the procedure has to re-read the same file several time.

% % % % % % % % % % % % %
\subsubsection{Conversions within R}
% % % % % % % % % % % % %



%%%%%%%%%%%%%%%%
%%%%%%%%%%%%%%%%
\section{Data analysis using \texttt{genlight} objects}
%%%%%%%%%%%%%%%%
%%%%%%%%%%%%%%%%

%%%%%%%%%%%%%%%%
\subsection{Principal Component Analysis (PCA)}
%%%%%%%%%%%%%%%%



%%%%%%%%%%%%%%%%
\subsection{Discriminant Analysis of Principal Components (DAPC)}
%%%%%%%%%%%%%%%%







\begin{thebibliography}{9}

\bibitem{tjart05}
  Jombart, T. (2008) adegenet: a R package for the multivariate
  analysis of genetic markers. \textit{Bioinformatics} 24: 1403-1405.

\bibitem{np145}
  R Development Core Team (2011). R: A language and environment for
  statistical computing. R Foundation for Statistical Computing,
  Vienna, Austria. ISBN 3-900051-07-0.

\end{thebibliography}


\end{document}
